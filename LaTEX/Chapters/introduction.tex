\chapter{Bevezetés}

A videójátékok az elmúlt években egy dollármilliárdos iparággá nőtték ki magukat. Nem csak a rohamosan fejlődő technológiának, de a szoftveres ágon tett előre lepéseknek is hála. A 2000-es évek elején a videójáték gyártás mindössze a programozók (matematikusok) számára egy hobbi volt, ami az akkori hardverből a lehető legtöbbet való kihozásról szólt. Manapság a játékmotoroknak hála bárki elkezdhet játékokat gyártani, viszont ez nem azt jelenti, hogy sokkal egyszerűbb lenne a videójáték gyártók dolga. A rohamosan fejlődő ipar és a játékfejlesztés elérhetőségének a növekedése az elvárásokat is megnövelte a játékok iránt.\\

Mivel a számítástechnika a matematikából nőtte ki magát, főleg a kezdetekben, ezért sok gyakorlatot alkalmazunk belőle. Egyike ezeknek a gyakorlatoknak az állapotgépek alkalmazása egyes esetekben. Név szerint, azon esetekben ahol fontos a könnyű bővíthetőség és az, hogy ne kelljen újabb változókat létrehozni annak érdekében, hogy kiküszöböljünk egyes eseteket. A legnépszerűbb példa egy állapotgép használatára az a videójátékokhoz kapcsolódik. A kezdetekben ez nem volt olyan fontos, mivel maguk a játékok egyszerűbbek voltak, lásd: Doom, Wolfenstein, Elite (1984) ahol a legfontosabb a grafika megoldása és helyes kirajzolása volt. Manapság, ahol ún. "Omni-Movement" van a játékokban, ami, ahogyan Charles\cite{charles2024} is megfogalmazta a posztjában, egy olyan mozgásfajta, ahol a játékos teste a fejétől függetlenül mozog, tehát a test és a fej (kamera) mutathat két különböző irányba. Ezen esetben a játékosoknak közelről kell figyelni a mozgásukat és cselekedeteiket, és jól definiált struktúrákat kell létrehozni egyes mozgássorozatoknak. A játékosok képesek különböző mozdulatok végrehajtására, többek között: futás, ugrás, csúszás és vetődés. Ezek mind precíz beviteleket követelnek meg a játékostól és az állapotok pontos nyilvántartását. Ennek a kivitelezéséhez az állapotgépek használata elengedhetetlen, mind a mozgásrendszer kivitelezéséhez, mind az animációk helyes lejátszásához.\\

A témámat, viszont nem az Omni-Movement inspirálta, az én projektemhez csak példaképpen kapcsolódik, hogy demonstráljam egy sokkal bonyolultabb koncepción is a témámat és megmutassam, hogy az iparban is használatos. Sokkal inkább a [Shibuya-punk esztétika](Shibuya-punk) királya és a mai napig a Sega egyik legkevésbé elismert kiadása a: Jet Set Radio (2000), és az az által inspirált Team Reptile játék a: Bomb Rush Cyberfunk (2023). Ezekről sokan nem hallottak, viszont, ha azt mondom, hogy "Tony Hawk Pro Skater", akkor az már több embernek fog ismerősen hangzani. A koncepció ugyan az mind a kettőnél: gurulj és csinálj menő trükköket (miközben próbálod magad nem összetörni, természetesen). Ha a felszínt nézzük is már eléggé bonyolultnak néz ki a helyzet, mind mozgás, mind pontszám számítás szempontjából; és elkezd  járni az agyunk azon, hogy mégis mennyi állapotot kellet nyomon követniük a fejlesztőknek, hogy ezeket megvalósítsák. Annak érdekében, hogy egy kicsit mélyebbre ássak és megválaszoljam ezt a kérdést úgy döntöttem, hogy magam is nekiállok és létrehozok egy mozgásrendszert, amiben tesztelhetjük, hogy mennyire is érné meg állapotgépeket használni, miért és mennyivel eredményezne szebb, jobban strukturált, átláthatóbb, könnyebben bővíthető kódot, és gyorsabb programot, mint egy hagyományos if-and megoldás.\\

A szakdolgozat végére szeretnék egy teljesen működőképes és játszható mozgásrendszer-prototípust elkészíteni, amely bemutatja az állapotgépek fontosságát a mozgásrendszerek kialakításában és a játékfejlesztés más terein. Mind ezt a Shibuya-punk esztétikában többnyire magam által elkészített modellekből.\\